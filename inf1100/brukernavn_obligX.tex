% REMEMBER TO SET LANGUAGE!
\documentclass[a4paper,10pt,english]{article}
\usepackage[utf8]{inputenc}
\usepackage[norsk]{babel}
% Standard stuff
\usepackage{amsmath,graphicx,varioref,verbatim,amsfonts,geometry}
% colors in text
\usepackage[usenames,dvipsnames,svgnames,table]{xcolor}
% Hyper refs
\usepackage[colorlinks]{hyperref}

% Document formatting
\setlength{\parindent}{0mm}
\setlength{\parskip}{1.5mm}

%Color scheme for listings
\usepackage{textcomp}
\definecolor{listinggray}{gray}{0.9}
\definecolor{lbcolor}{rgb}{0.9,0.9,0.9}

%Listings configuration
\usepackage{listings}
%Hvis du bruker noe annet enn python, endre det her for å få riktig highlighting.
\lstset{
	backgroundcolor=\color{lbcolor},
	tabsize=4,
	rulecolor=,
	language=python,
        basicstyle=\scriptsize,
        upquote=true,
        aboveskip={1.5\baselineskip},
        columns=fixed,
	numbers=left,
        showstringspaces=false,
        extendedchars=true,
        breaklines=true,
        prebreak = \raisebox{0ex}[0ex][0ex]{\ensuremath{\hookleftarrow}},
        frame=single,
        showtabs=false,
        showspaces=false,
        showstringspaces=false,
        identifierstyle=\ttfamily,
        keywordstyle=\color[rgb]{0,0,1},
        commentstyle=\color[rgb]{0.133,0.545,0.133},
        stringstyle=\color[rgb]{0.627,0.126,0.941}
        }
        
\newcounter{subproject}
\renewcommand{\thesubproject}{\alph{subproject}}
\newenvironment{subproj}{
\begin{description}
\item[\refstepcounter{subproject}(\thesubproject)]
}{\end{description}}

%Lettering instead of numbering in different layers
%\renewcommand{\labelenumi}{\alph{enumi}}
%\renewcommand{\thesubsection}{\alph{subsection}}

%opening
\title{FYS-MEK1110 - Oblig X}
\author{Arne Bjarne}

\begin{document}

\maketitle

\section{Oppgave a}

Newtons andre lov sier at ved konstant masse er kraft proporsjonal med akselerasjon. 
\[ F = ma \]
Eller på vektorform
\[ \vec F = m \vec a \]
Noen ganger er det kjekt å inkludere formler i teksten. 
Det gjør vi slik: $F = ma$. 

Dersom du trenger hjelp til å skrive en formel, anbefales \url{https://www.codecogs.com/latex/eqneditor.php} hvor du kan trykke inn formelen du vil ha og få produsert latexkode. 
Jo flere formler du skriver, jo raskere det går det og jo sjeldnere trenger du å slå opp. 

\subsection{Underkategori}

Av og til vil vi dele opp oppgavene enda litt mer. 


\section{Bilder}

Denne tekstsnippen gjør det enkelt å inkludere bilder 

\begin{figure}[h!]
        \centering 
        %Scale angir størrelsen på bildet. Bildefilen må ligge i samme mappe som tex-filen. 
        \includegraphics[scale=0.2]{eksempel_graf.jpg} 
        \caption{Bildetekst. Dette er en kul graf som viser det den skal.}
        %Label gjør det enkelt å referere til ulike bilder.
        \label{fig:eksempelbilde1}
\end{figure}

%Referanser fungerer først når du har kompilert to ganger!
Denne figure kan jeg nå referere til ved å skrive: Figur \ref{fig:eksempelbilde1}. 


\section{Kode}

Det er ofte fint å inkludere kode i programmet også. 
Denne tekstsnippen gjør det:

\lstinputlisting{eksempel_kode.py}

Dersom du skriver i matlab kan du få annen highlighting ved å endre ''lstset'' lenger opp i dokumentet. 


\end{document}
